\section{Related Work}
\subsection*{Nearest neighbor based complexity measures}
Most of the performant dataset complexity measures in the imbalanced classification problem space is based on nearest neighbors. 
In Barella et al. \cite{DBLP:journals/isci/BarellaGSLC21} \emph{Assessing the data complexity of imbalanced datasets} they adapt a selection of existing data complexity measures, to account for class imbalance. Through their very thorough experimental framework, they find some extremely robust measures where possibly the two most noticeable measures are their adaptation of \emph{N1} and \emph{N3}.
\begin{equation}
    \label{eq:n1}
     \text{N1}_{c_1}(T)=\frac{1}{n_{c_{1}}}\sum_{i=1}^{n_{c_{1}}}\text{I}((x_i^{c_1},x_j)\in \text{MST} \wedge c_1 \neq y_j)
\end{equation}
The adapted \emph{N1} measure as described in \ref{eq:n1} builds a minimum spanning tree (MST) over the dataset, and returns the fraction of edges from class from vertices belonging to class $c_1$ are connected  to vertices of class $y_j$. The identity function $\text{I}$ in the equation is 1 if the inner expression is true.

\begin{equation}
    \label{eq:n3}
    \text{N3}_{c_1}(T)=\frac{\sum_{i=1}^{n}\text{I}(\text{NN}(x_i^{c_1}) \neq c_1)}{n_{c_{1}}}
\end{equation} The adapted \emph{N3} measure returns the leave one out error rate of the first nearest neighbor classifier for class $c_1$. Here $\text{NN}(x_i)$ returns the label of the nearest neighbor of instance $x_i$. 

For both the adapted \emph{N1} and \emph{N3} which are adapted to be class specific measures, the aggregate measure for the entire dataset is found by taking the mean of each measure. 

Mercier et al. \cite{DBLP:conf/ida/MercierSASSS18} presents a measure dubbed Degree IR (degIR) this measure measures the complexity as a function of local imbalance and overlap. The overlap is determined by the $5\text{NN}$ set. If the 5-nearest neighbors of an instance belong to the same class as the instance, then the instance is not a region of overlap. Otherwise it is.  

The number of points from the minority class, that are in an overlapping region is defined as $n_{min\_over}$ and the number of points from the majority class, that are in an overlapping region is defined as $n_{maj\_over}$. Then the overlap degree is defined as shown in Equation \ref{eq:overlap degree}
\begin{equation}
\label{eq:overlap degree}
\frac{n_{min\_over} + n_{maj\_over}}{n}
\end{equation}
The degree IR after normalization is defined as \ref{eq:degIR}
\begin{equation}
\label{eq:degIR}
1-\frac{n_{min\_over}}{\frac{n}{2}}=1-\frac{2n_{min\_over}}{n}
\end{equation}
